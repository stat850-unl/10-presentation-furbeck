\documentclass{tccv}
\usepackage[english]{babel}

\begin{document}

\part{Rebecca Furbeck}
\begin{eventlist}
\section{Current research emphasis}
\item{Summer 2018 - Current}
     {Meat and Muscle Biology}
     {University of Nebraska - Lincoln}

Ph.D. candidate with research emphasis
regarding spoilage dynamics and microbial
ecology of

processed turkey throughout shelf life.
Studies

include utilization of high throughput 16S rRNA
sequencing and traditional plate methods to
investigate impact of product formulation,
processing steps, packaging, and manufacturing location on microbial ecology. Data will provide insight on bacteria responsible for spoilage, and potential methods of inhibition for shelf life extension.

\vspace{2mm}
Other projects include both industry and interdepartmental collaboration in the following areas:

Natural ingredient utilization, sous-vide,  swine diet impact on pork quality, sensory evaluation of meats, bison product development, microbial ecology of dry-aged beef, process optimization, and microbial process lethality.


\section{Industry experience}
\item{Summer 2017}
     {Research and Development Intern}
     {Hormel Foods}

Conducted high-pressure processing (HPP) efficacy challenge study on reconstructed deli-meats, and applied study results to make suggestions for

formula optimization. Formulated cost-reducing spice blends for dry sausage application. Assisted product development scientists and thermal processing authorities on line-trials of both meat and grocery products in production facilities.

\item{Summer 2016}
     {Research and Development Intern}
     {West Liberty Foods}

Researched USDA Organic compliance and
developed reconstructed deli meat formulas.
Resolved formula issues regarding protein
extraction, cook loss, and antimicrobial application. Conducted sensory panels and statistical analysis of collected data. Gathered nutritional specifications to update \\Nutrition Facts panel.
\end{eventlist}

\personal
    []
    {213A Animal Science

    Lincoln, NE 68583-0908}
    {+1-309-531-6017}
    {rfurbeck@huskers.unl.edu}

\section{Education}

\textbf{Meat Science, Graduate Research Assistant}

{  University of Nebraska - Lincoln, Ph.D.}

{Anticipated Graduation Summer 2022}

{Advisor - Dr.Gary Sullivan}

{GPA 3.818/4.00}


\vspace{2mm}

    \textbf{B.S. Food Science and Technology}

{  Iowa State University, May 2018}

{Minor: Applied Statistics}

{GPA 3.59/4.00}

\section{Organization membership}

\begin{yearlist}

\item{2018-2020}
    {Animal Science Graduate

    Student Association}
    {Vice President}

\item{2018-2020}
    {UNL Meat Science Club}
    {President}

\item{2017-2020}
     {American Meat Science

     Association}
     {Graduate Student Member}

     \item{2014-2019}
     {Institute of Food Technologists}
     {Iowa and Student Sections}







\end{yearlist}



\section{Teaching Experience}
\begin{eventlist}

\item{Spring 2019}
     {Graduate Teaching Assistant}
     {AN SCI 410 - Processed Meats}

Facilitated lecture and laboratory portions of course to enhance student understanding of meats processing, and assisted their utilization of this knowledge through product development projects. Lecture focuses include: fermentation, microbial safety, and spoilage.

\item{Spring 2018}
     {Undergraduate Teaching Assistant}
     {Food Microbiology Laboratory}


Developed laboratory handout teaching materials to teach students statistical analysis methods for microbial enumeration studies.
\end{eventlist}

\newpage
\begin{eventlist}
\section{Extension Activities}

\item{}
    {Meat Lab Production - Spring 2020}
     {NAMP Short Course}

\\
Assisted in producing injected/enhanced products (pressed bacon, marinated poultry, papain tenderized beef chuck) to inform the Nebraska Association of Meat Processor's members on current value-addition methodology and ingredients.





\item{}
    {Laboratory Management and Catering - Spring 2019}
     {U.S. Meat Export Federation Short Course}

\\

Pre-prepared samples and demonstrated

production of frankfurters, smoked sausage,

bacon, reconstructed hams, pulled pork, injected pork loin, and strip loin roasts under USDA

inspection. Provided technical information

regarding product formulation and manufacturing techniques currently utilized in the U.S. for

international industry personnel.

\item{}
    {Event Assistant - Spring 2019 and Spring 2020}
     {Nebraska Youth Beef
     Leadership \\Symposium}

\\
Provided laboratory/kitchen assistance for students during culinary development portion of the

conference. Enhanced student understanding of beef merchandising, cookery, safe food handling, and recipe development.

\item{}
    {Videographer - Spring 2018}
     {Iowa State University Bacon Expo}

\\

Created "Makin' Bacon at ISU" video in conjunction with ISU Meat Extension and Meat Laboratory staff to educate Iowa State Bacon Expo attendees about bacon processing and ingredient roles in formulation.



\end{eventlist}

\section{Software skills}

\begin{factlist}

\item{APPLIED}
     {R (DADA2), PICRUSt, STAMP Bioinformatics, QIIME}

\item{GENERAL}
     { \LaTeX, Genesis for Research and Development, SAS , JMP, Infor Optiva, Exponent (Stable Microsystems), RedJade}


\end{factlist}


\section{Certifications}
\begin{yearlist}

\item{2020}
    {HACCP Certification}

\item{2017}
     {FSMA Preventative Controls

     Qualified Individual}



\end{yearlist}

\vspace{500}
\section{Leadership Experience}
\begin{eventlist}

\item{Fall 2018 - Current}
     {University of Nebraska-Lincoln}
     {Meat Science Club, President}

Created UNL Meat Science Club, an East Campus registered student organization, operating to enhance student opportunity and involvement within meat science research and industry. Developed club constitution, arranged industry speakers for club presentations, and promoted club events.

\item{Spring 2016 - Spring 2018}
     {ISU Food Science Club}
     {Public Relations Representative}

Educated underclassmen about career options in the food industry by conducting interviews and writing articles on  student's summer internship experiences. Helped organize "Fields of Opportunities" food science networking event with industry professionals.
\end{eventlist}



\section{Publications and Presentations}
\item{Furbeck RA, Fernando SC, Sullivan GA. 2020.

\textbf{Effect of processing parameters and storage time on the spoilage microbiome of turkey products}. Poster presented at: 66th International Congress of Meat Science and Technology and 73rd Reciprocal Meat Conference (ICoMST/RMC); 2020 Aug 3-6, virtual meeting.  }

\item
Watson SC, Furbeck RA, Chaves BD, Sullivan GA. \textbf{Spoilage \textit{Pseudomonas} survive thermal processing and grow during vacuum packaged storage in an emulsified beef system}. Poster presented at: 66th International Congress of Meat Science and Technology and 73rd Reciprocal Meat Conference (ICoMST/RMC); 2020 Aug 3-6, virtual meeting.


\item{Furbeck RA, Knapp JP, Trenhaile-Grannemann MD, Burkey TE, Fernando SC, Sullivan GA, Miller PS. \textbf{Effect of Green Grass supplementation of swine diet on pork quality characteristics}.  University of Nebraska-Lincoln. Poster presented at: Midwest American Society of Animal Science Conference; 2020 Mar 2; Omaha, NE. }



\item{Furbeck RA and Sullivan GA. \textbf{"From Bellies to Bacon."} \textit{Nebraska Pork Talk}, July 2018.}





\begin{center}
\line(1,0){230}
\end{center}
\centering
\textit{References available upon request.}

\end{document}



